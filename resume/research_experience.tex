\pagebreak

\cvsection{Research Experience}

\begin{cventries}
  \cventry
    {Research Associate} % Job title
    {Louisiana State University} % Organization
    {Molecular Simulation} % Location
    {2007/11 - present} % Date(s)
    {
      \begin{cvitems} % Description(s) of tasks/responsibilities
        \item {\underline{Function and dynamics of the enzyme biotin carboxylase}}
            \begin{itemize}
            \item {Showed that the ATP binding domain of biotin carboxylase (BC) is most stable in a closed configuration for a monomer and most stable in an open configuration for a dimer which may explain the loss of activity for monomers relative to dimers}
            \item {Showed that there is communication between monomers of a BC dimer when ATP is bound in one monomer, and that the most stable state in this case is one with both monomers closed, supporting other evidence suggesting that a reaction occurs in only one monomer at a time}
            \end{itemize}
        \item {\underline{Sequencing DNA using times-of-flight of single nucleotides as they pass through nanochannels}}
            \begin{itemize}
            \item {Performed initial simulations in nanoslits. Slit walls were composed of a single type of atom in a disordered arrangement approximating polymethyl methacrylate (PMMA). Smooth and rough walls were considered.}
            \item {Assisted graduate students with running simulations with slit walls composed of PMMA and self-assembled monolayers.}
            \item {Extrapolated simulation results to estimate the slit length required to obtain a given error rate as a metric to compare different surfaces.}
            \end{itemize}        
        \item {\underline{Bulk and interfacial properties of liquid metals \& alloys}}
        \item {\underline{Parameterization of phase field models for rapid solidification of Ti \& Ti-Ni alloys using molecular dynamics simulations}}
            \begin{itemize}
            \item {Performed simulations to obtain kinetic coefficients for Ti. Assisted graduate student with setup and analysis of other kinetic coefficient and interfacial free energy simulations.}
            \item {Wrote python code to analyze solid-liquid interfacial properties including velocity, interfacial free energy, and concentration profiles.}
            \end{itemize}
        \item {\underline{Growth of Cu on TiN}}
            \begin{itemize}
            \item {Assisted graduate student with setup of simulations which showed that during deposition of Cu on TiN(100), bcc phase grows initially followed by martensitic transformation of bcc to twinned fcc(110)}
            \end{itemize}
        \item {\underline{Self-assembly of bile salts or cationic linear peptide analogs (LPAs) and their interaction with lipid bilayers}}
            \begin{itemize}
            \item {Assisted graduate student and advisor with running bile salt simulations}
            \item {Showed that LPAs with hydrocarbon connectors longer than 7 carbons form small, stable micelles at high concentrations.}
            \item {Showed that longer LPA hydrocarbon connectors lead to deeper penetration into lipid bilayers.}
            \item {Observed the existence of a long-lived trans-bilayer configuration for LPAs with 11 carbon hydrocarbon connectors.}
            \end{itemize}
        \item {\underline{Behavior of alpha-tocopherol (vitamin E) in lipid bilayers}}
            \begin{itemize}
            \item {Performed initial simulations. Assisted graduate student with further simulations.}
            \item {Analyzed trajectory data to show that the higher flip-flop rate of tocopherol at higher concentrations in thinner bilayers such as DMPC is associated with tocopherol clusters that span the bilayer.}
            \end{itemize}
        \item {\underline{Interaction of lignin oligomers with lipid bilayers or β-cyclodextrin or microwaves}}
            \begin{itemize}
            \item {Assisted graduate student with the design and analysis of the simulations.}
            \item {Suggested a normalized deuterium order parameter averaged over several carbon atoms in the lipid tails for monitoring the lipid gel-liquid crystalline phase transition.}
            \end{itemize}            
        \item {\underline{Interaction of poly(lactic-co-glycolic acid) (PLGA) nanoparticles with lipid bilayers}}
        \item {\underline{Effect of DMSO on lipid bilayers}}
        \item {\underline{Self-assembly of VECAR (vitamin E-carnosine) bolaamphiphiles}}
            \begin{itemize}
            \item {Assisted collaborators at Southeastern Louisiana University with simulation setup and analysis.}
            \end{itemize}
        \item {\underline{Hybrid continuum/molecular dynamics simulations}}
            \begin{itemize}
            \item {Assisted graduate student with setup and analysis of simulations.}
            \end{itemize}
        \item {\underline{Melting point estimation using unsupervised machine learning}}
            \begin{itemize}
            \item {Assisted collaborators in Physics Department with setup and analysis of simulations.}
            \end{itemize}      \end{cvitems}
    }
    
  \cventry
    {Graduate Research Assistant} % Job title
    {University of Notre Dame} % Organization
    {Chemical Engineering} % Location
    {2002/08 - 2007/09} % Date(s)
    {
      \begin{cvitems} % Description(s) of tasks/responsibilities
        \item {Showed that 4.5 nm indentations cause a large increase in nucleation rate relative to a smooth surface, but 1.5 nm indentations have little effect
        \item {Applied a thermodynamic model using Mathematica that explained the location of critical bubble nuclei near flat surfaces and extended this model to geometric surface defects}
}
      \end{cvitems}
    }
    
  \cventry
    {Undergraduate Research Assistant} % Job title
    {Kansas State University} % Organization
    {Chemistry} % Location
    {1999/01 - 2001/01} % Date(s)
    {
      \begin{cvitems} % Description(s) of tasks/responsibilities
        \item {Showed the effects of solute-gel interactions by measuring electrochemiluminescence intensities of Ru(bpy)\textsubscript{3}\textsuperscript{2+} and diffusion coefficients of Co(bpy)\textsubscript{3}\textsuperscript{2+} in silica gels and organically modified silica gels}
      \end{cvitems}
    }
    
\end{cventries}
