\pagebreak

\cvsection{Research Experience}

\begin{cventries}
  \cventry
    {Research Associate} % Job title
    {Louisiana State University} % Organization
    {Molecular Simulation} % Location
    {2007/11 - 2021/10} % Date(s)
    {
        \begin{cvitems} % Description(s) of tasks/responsibilities
            \item {\textbf{\underline{Function and dynamics of the enzyme biotin carboxylase}}}
                \vspace{2pt}
                \begin{itemize}
                \item {Estimated free energy differences for opening ATP binding domains for dimers or monomers with or without bound ATP to support experimental observations of monomer inactivity and dimer half-sites reactivity.}
                \end{itemize}
            \item {\textbf{\underline{Sequencing DNA using times-of-flight of single nucleotides}}}
                \vspace{2pt}
                \begin{itemize}
                    \item {Determined the length of a slit necessary to achieve a particular sequencing error rate using a statistical analysis of extrapolated simulation data allowing various surfaces to be compared.}
                    \item {Simulated 3 nanometer wide slits with walls composed of different self-assembled monolayers which showed that the surface chemistry can have a large effect on sequencing efficiency.}
                \end{itemize}        
            \item {\textbf{\underline{Self-assembly of cationic linear peptide analogs (LPAs) and their interaction with lipid bilayers}}}
                \vspace{2pt}
                \begin{itemize}
                \item {Showed that LPAs with hydrocarbon connectors longer than 7 carbons form small, stable micelles at high concentrations.}
                \item {Showed that longer LPA hydrocarbon connectors lead to deeper penetration into lipid bilayers.}
                \item {Observed the existence of a long-lived trans-bilayer configuration for LPAs with 11 carbon hydrocarbon connectors.}
                \end{itemize}
            \item {\textbf{\underline{Behavior of alpha-tocopherol (vitamin E) in lipid bilayers}}}
                \vspace{2pt}
                \begin{itemize}
                \item {Designed simulations to determine the most probable depth of the tocopherol chromanol groups in several types of lipid bilayers.}
                \item {Used DBSCAN clustering to show that higher rates of tocopherol flipping between bilayer leaflets at high concentrations in thinner bilayers are associated with bilayer spanning tocopherol clusters.}
                \end{itemize}
            \item {\textbf{\underline{Interaction of lignin dimers with lipid bilayers or β-cyclodextrin}}}
                \vspace{2pt}
                \begin{itemize}
                    \item {Captured lipid transition temperature by defining and analyzing a normalized deuterium order parameter of the lipid tails.}
                    \item {Utilized unsupervised machine learning to estimate the proportions of different lignin-cyclodextrin bound states.}
                \end{itemize}            
            \item {\textbf{\underline{Interaction of lignin tetramers with microwaves}}}
                \vspace{2pt}
                \begin{itemize}
                    \item {Showed that some bonds for lignin in deep eutectic solvents are more susceptible to microwaves including the most common monomer linkage.}
                \end{itemize}   
            % \item {\textbf{\underline{Interaction of poly(lactic-co-glycolic acid) (PLGA) nanoparticles with lipid bilayers}}}
            % \item {span-80}
            \item {\textbf{\underline{Effect of DMSO on lipid bilayers}}}
                \vspace{2pt}
                \begin{itemize}
                    \item Simulations of DMPC bilayers with and without added DMSO showed that DMSO increased the amount of water in the bilayer and rate of permeation through the bilayer, changed the lipid head group orientations and dyanmics, and modified the electrostatic potential profile across the bilayer.
                \end{itemize}
            \item {\textbf{\underline{Self-assembly of VECAR (vitamin E-carnosine) bolaamphiphiles}}}
                \vspace{2pt}
                \begin{itemize}
                \item {Assisted collaborators at Southeastern Louisiana University with simulation setup and analysis.}
                \item{Assembly was more elongated \& branched and formed faster with decreasing acyl chain length between the vitamin E \& carnosine ends.}
                \end{itemize}
            \item {\textbf{\underline{Bulk and interfacial properties of molten metals \& alloys}}}
                \vspace{2pt}
                \begin{itemize}
                    \item Computed cohesive energies, densities, diffusivities, pair distribution functions, viscosities, surface tensions and their temperature dependences near the liquidus temperatures for molten Al, Cr, Ni, Ti, Al-Ni, Al-Ti, and Cr-Ni with assistance from REU students.
                \end{itemize}
            \item {\textbf{\underline{Parameterization of phase field models for solidification using molecular dynamics simulations}}}
                \vspace{2pt}
                \begin{itemize}
                    \item {Designed an algorithm to compute kinetic coefficients of alloys entirely from simulations for the first time.}
                    \item {Computed kinetic coefficients for alloys entirely from simulations for the first time.}
                    \item {Developed software to analyze solid-liquid interfacial properties, including velocity, interfacial free energy, and concentration profiles.}
                \end{itemize}
            \item {\textbf{\underline{Growth of Cu on TiN}}}
                \vspace{2pt}
                \begin{itemize}
                \item {Simulations which showed that during deposition of Cu on TiN(100), bcc phase grows initially followed by martensitic transformation of bcc to twinned fcc(110) consistent with experiments.}
                \end{itemize}
            \item {\textbf{\underline{Melting point estimation using unsupervised machine learning}}}
                \vspace{2pt}
                \begin{itemize}
                \item {Assisted collaborators in Physics Department with setup and analysis of simulations.}
                \end{itemize}
            \item {\textbf{\underline{Hybrid continuum/molecular dynamics simulations}}}
                \vspace{2pt}
                \begin{itemize}
                \item {Applied the method to two-phase liquid systems including a realistic system of hexane/water in contact with a PMMA surface.}
                \end{itemize}
        \end{cvitems}
    }

  \cventry
    {Graduate Research Assistant} % Job title
    {University of Notre Dame} % Organization
    {Chemical Engineering} % Location
    {2002/08 - 2007/09} % Date(s)
    {
      \begin{cvitems} % Description(s) of tasks/responsibilities
        \item {\textbf{\underline{Bubble nucleation}}}
            \begin{itemize}
                \item {Showed that 4.5 nm indentations cause a large increase in nucleation rate relative to a smooth surface, but 1.5 nm indentations have little effect}
                \item {Applied a thermodynamic model using Mathematica that explained the location of critical bubble nuclei near flat surfaces and extended this model to geometric surface defects}
            \end{itemize}
      \end{cvitems}
    }
    
  \cventry
    {Undergraduate Research Assistant} % Job title
    {Kansas State University} % Organization
    {Chemistry} % Location
    {1999/01 - 2001/01} % Date(s)
    {
      \begin{cvitems} % Description(s) of tasks/responsibilities
        \item {\textbf{\underline{Electrochemistry in silica gels}}}
            \begin{itemize}
                \item {Showed the effects of solute-gel interactions by measuring electrochemiluminescence intensities of Ru(bpy)\textsubscript{3}\textsuperscript{2+} and diffusion coefficients of Co(bpy)\textsubscript{3}\textsuperscript{2+} in silica gels and organically modified silica gels}
            \end{itemize}
      \end{cvitems}
    }
    
\end{cventries}
